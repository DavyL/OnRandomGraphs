\documentclass{beamer}
 \usepackage{amsmath, amssymb, amsthm, bbm}

\usepackage{tikz}
\usepackage{tkz-berge}
\usetikzlibrary{positioning}
\usepackage[position=top]{subfig}
\usetikzlibrary{arrows, positioning, automata}

%\theoremstyle{definition}
%\newtheorem{definition}{Definition}[section]


\usepackage[utf8]{inputenc}
\usetheme{Berlin}
\title[]{Random graphs}
 
%\subtitle{The Erd\H{o}s-R\'enyi model}
 
\author[]{Leo Davy}
 
\date[]{Toulouse, May 27th 2019}

\begin{document}

	\frame{\titlepage}

	\begin{frame}
		\begin{definition}
			A graph is a pair $G = (V,E)$:
			\begin{itemize}
				\item $V$ is the vertex set
				\item $E$ is the edge set, $E \subseteq \{\{x,y\}: x,y \in V\}$
			\end{itemize}
		\end{definition}
		Interest :
		\begin{itemize}
			\item Simple graphical representation
			\item Allows to describe phenomenas
			\item and much more
		\end{itemize}
	\end{frame}
	
	\begin{frame}
		\frametitle{An example of real world graph}
		A collaboration graph:
		\begin{itemize}
			\item $V$ the set of scientists
			\item $E = \{\{x,y\}: x $ and $y$ co-authored a paper$\}$
		\end{itemize}
		Experimental results:
		\begin{itemize}
			\item Positive density of triangles
			\item Degree distribution follows a power law 
		\end{itemize}
		\begin{block}{Remark}
			The Erd\H{o}s collaboration graph is the subgraph of the above collaboration graph s.t. $d($Erd\H{o}s$, x) < \infty$ for all $v\in V$.
		\end{block}
	\end{frame}

	\begin{frame}
		\frametitle{Another example of real world graph}
		The Web-graph:
		\begin{itemize}
			\item $V$ the set of web pages
			\item $E = \{(x,y):x,y \in V, x$ contains a link to $y$\}
		\end{itemize}
		\begin{itemize}
			\item The graph is directed
			\item The degrees follow a power law
		\end{itemize}
	\end{frame}

	\begin{frame}
		\begin{definition}
			A random graph is a graph where nodes, edges are both are selected by a random procedure.
		\end{definition}
		Restrict ourselves to \emph{static} random graphs on $n$ vertices.
		\begin{block}{The model:}
			\begin{itemize}
				\item $V$ is fixed ($V = [n] = \{1,2,\ldots,n\}$).
				\item No loops.
				\item No parallel edges.
				\item Total number of possible edges : $N= \binom{n}{2} = \frac{n(n-1)}{2}$.
			\end{itemize}
		\end{block}
	\end{frame}

	\begin{frame}
		\frametitle{The Erd\H{o}s-R\'enyi model}
		\begin{definition}[ER 1960]
			$\mathcal{G}_{n,M}$ : A graph is picked uniformly at random among all graphs with $M$ edges on $n$ vertices.
		\end{definition}
		Let $G$ a graph with $M$ edges and $n$ vertices, then
		\begin{equation}
			\mathbb{P}_M (G) = 1/\binom{N}{M}
		\end{equation}
	\end{frame}
	\begin{frame}
		\frametitle{The Erd\H{o}s-R\'enyi model}
		\begin{definition}[Gilbert 1959]
			$\mathcal{G}_{n,p}$ : Each of the $N = \binom{n}{2}$ possible edges is chosen independently with probability $p \in [0,1]$.
		\end{definition}
		Let $G$ a graph with $n$ vertices, then
		\begin{equation}
			\mathbb{P}_p (G) = p^{e_G}(1-p)^{N-e_G}
		\end{equation}
	\end{frame}
	\begin{frame}
		\frametitle{The stability number}
		\begin{definition}
			$\alpha(G) :=$ largest set of vertices s.t. no vertices are adjacent.
		\end{definition}
		\begin{theorem}
			$\alpha(G \in \mathcal{G}_{n,p}) \leq \lceil 2p^{-1}\log n\rceil$ w.h.p.
		\end{theorem}
	\end{frame}
	\begin{frame}
		Let $S \subseteq V$, $|S| = k+1$,
		\begin{equation}
			\mathbb{P}("\text{S is a stable set}") = (1-p)^{\binom{k+1}{2}}
		\end{equation}
		Let $X_{k+1} = \sum_{S} \mathbbm{1}("\text{S is a stable set}")$.
		\begin{block}{Remark}
			We have $X_k = 0$ if $k> \alpha$.
		\end{block}
		\begin{equation}
			\mathbb{E}X_{k+1} = \binom{n}{k+1} (1-p)^{\binom{k+1}{2}} \leq \frac{(ne^{-p\frac{k}{2}})^{k+1}}{(k+1)!}
		\end{equation}
		The RHS goes to 0 if $k \geq 2p^{-1}\log n$.
	\end{frame}
	\begin{frame}
		\frametitle{Is $\mathcal{G}_{n,p}$ connected ?}
		\begin{definition}
			A graph is connected if there is a path linking any two vertices.
		\end{definition}
		\begin{theorem}
			Let $p = \frac{\log n + c}{n}$ with $c\in \mathbb{R}$, then
			$\mathbb{P}(G \in \mathcal{G}_{n,p}\text{ is connected } )\longrightarrow_n e^{-e^{-c}}$.
		\end{theorem}
		\begin{block}{Remark}
			Let $\epsilon >0$
			\begin{itemize}
				\item If $p = (1+\epsilon) \frac{\log n}{n}$, then $\mathcal{G}_{n,p}$ is connected w.h.p. 
				\item If $p = (1-\epsilon) \frac{\log n}{n}$, then $\mathcal{G}_{n,p}$ is disconnected w.h.p. 
			\end{itemize}
		\end{block}
	\end{frame}
	\begin{frame}	
		\frametitle{Is $\mathcal{G}_{n,p}$ connected ?}
		\begin{definition}
			A graph is connected if there is a path linking any two vertices.
		\end{definition}
		\begin{theorem}
			Let $p = \frac{\log n + c}{n}$ with $c\in \mathbb{R}$, then
			$\mathbb{P}(G \in \mathcal{G}_{n,p}\text{ is connected } )\longrightarrow_n e^{-e^{-c}}$.
		\end{theorem}

		In order to prove this result, we need to know the probabilities that $\mathcal{G}_{n,p}$ contains :
		\begin{itemize}
			\item Isolated vertex
			\item Isolated edges
			\item No isolated component of size between 3 and $\lceil \frac{n}{2} \rceil$
		\end{itemize}
	\end{frame}
	\begin{frame}
		$X_k$ : number of connected component of size $k$.
		\newline
		Does $\mathcal{G}_{n,p}$ contain an isolated edge ? 
		\begin{align}
			\mathbb{P}(X_2 \geq 1) \leq \mathbb{E} X_2 	&= \binom{n}{2}p((1-p)^{n-1})^2\\
									&= \mathcal{O}(n^2 p e^{-2pn}) = \mathcal{O}(p) = o(1).
		\end{align}
		\begin{block}{Result}
			No isolated edge with high probability.
		\end{block}
	\end{frame}
	\begin{frame}
		Does $\mathcal{G}_{n,p}$ contain an isolated component of size $3 \leq k \leq n/2$ ? 
		First moment method requires the number of connected components of size $k$.
		\begin{block}{Remark}
			Any connected component on $k$ vertices contains a tree on $k$ vertices. 
		\end{block}
		\begin{theorem}{Cayley's formula}
			There are $k^{k-2}$ possible spanning trees on $k$ vertices.
		\end{theorem}
	\end{frame}
	\begin{frame}
		Let $q_k$ the probability that a set of $k$ vertices doesn't connect with any other vertex outside the set.
		\begin{align}
			\mathbb{P}(X_k \geq 1) \leq \mathbb{E} X_k 	&\leq \binom{n}{k}k^{k-2}q_k\\
									&= p^{-1}(\mathcal{O}(\frac{\log n}{\sqrt{n}}))^k
		\end{align}
		\begin{equation}
			\sum_{k=2}^{\lceil \frac{n}{2} \rceil} \mathbb{P}(X_k \geq 1) \leq o(1) + p^{-1}\sum_{k=3}^{\lceil \frac{n}{2} \rceil} A^k \longrightarrow 0
		\end{equation}
		With high probability $\mathcal{G}_{n,p}$ doesn't contain isolated component larger than 1.
	\end{frame}
	\begin{frame}
		What is the distribution of isolated vertices ?
		\begin{theorem}
			In $\mathcal{G}_{n,p}$ the number of isolated vertices follows a Poisson distribution of mean $e^{-c}$.
		\end{theorem}
		Then, $\mathbb{P}(X_1 = 0) = e^{-e^{-c}}$.
		\newline
		The probability that $\mathcal{G}_{n,p}$ is connected tends to $e^{-e^{-c}}$.
	\end{frame}
	\begin{frame}
		\titleframe{The phase transition}
		What does $\mathcal{G}_{n,\frac{\lambda}{n}}$ look like ?
		\begin{block}{The subcritical phase: $\lambda < 1$}
			\begin{itemize}
				\item Tree like components (well approximated by a GW process).
				\item Largest component of size $\mathcal{O}(\log n)$.
			\end{itemize}
		\end{block}
		\begin{block}{The supercritical phase: $\lambda>1$}
			\begin{itemize}
				\item A single complex Giant Component of size $\theta(n)$.
				\item All other components are simple and of size $\mathcal{O}(\log n)$.
			\end{itemize}
		\end{block}
	\end{frame}
	


\end{document}

